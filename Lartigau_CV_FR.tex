\documentclass[a4paper, 10pt]{article}
\usepackage{Baplar-CV}

\usepackage{hyperref}
\usepackage{fontawesome5}
\usepackage{fontspec}
\setmainfont[Ligatures=TeX]{Noto Serif}
\setsansfont[Ligatures=TeX]{Noto Sans}
\setmonofont[Ligatures=TeX]{Noto Sans Mono}
\renewcommand{\familydefault}{\sfdefault}

\usepackage{polyglossia}
\setmainlanguage{french}

\author{Baptiste LARTIGAU}
\title{Ingénieur Logiciel et DevOps}

\begin{document}
\thispagestyle{empty}
\maketitle

\contact{\faPhone}{(+33) 06 45 64 68 70}
\contact{\faEnvelope[regular]}{\href{mailto:baptiste.lartigau@gmail.com}{\texttt{baptiste.lartigau@gmail.com}}}
\contact{\faLinkedin}{\url{www.linkedin.com/in/baplar}}


\section{\faBlackTie}{Expérience}

\begin{datedEntry}
    {Jan. 2023 -- Auj.}
    {\textbf{Ingénieur logiciel} chez \textbf{Shippingbo} -- Toulouse, France}
    \entryItem{Développement d’un système de gestion logistique tout-en-un en \textbf{Ruby on Rails}}
    \entryItem{Création et maintenance de connecteurs API (REST, SOAP) pour services transporteurs}
    \entryItem{Activités \textbf{DevOps}, amélioration des environnements de build et développement}
\end{datedEntry}

\begin{datedEntry}
    {Sep. 2019 -- Dec. 2022}
    {\textbf{Ingénieur logiciel} et \textbf{Tech lead} chez \textbf{SII} -- Toulouse, France}
    Mission chez \textbf{THALES ALENIA SPACE} pour le projet SpaceOps

    \smallskip
    % \entryItem{\textbf{Tech Lead} de l’une des squads de développement du projet}
    \entryItem{Développement de fonctionnalités sur de grandes bases de code \textbf{Java}}
    \entryItem{Responsable \textbf{DevOps}, amélioration et optimisation des pipelines de build et déploiement}
\end{datedEntry}

\begin{datedEntry}
    {Oct. 2018 -- Mai 2019}
    {\textbf{Ingénieur Machine Learning} chez \textbf{COLLINS AEROSPACE} -- Toulouse, France}
    % \entryItem{Étude de l’état de l’art des modèles de ML pour la détection d’anomalies}
    \entryItem{Implémentation du modèle bayésien \textbf{HDP-HMM} en \textbf{Python / NumPy}}
    \entryItem{Conception d’une solution complète de détection d’anomalies pour logs IMO A350}
\end{datedEntry}

\iffalse
\begin{datedEntry}
    {Juil. -- Oct. 2018}
    {\textbf{Ingénieur sécurité cloud AWS} chez \textbf{AIRBUS} -- Toulouse, France}
    \entryItem{Application du \textbf{AWS Well Architected Framework} aux infrastructures d’Airbus}
    % \entryItem{Évaluation d’outils de contrôle de conformité (AWS Config, Cloud Custodian, Evident.io)}
    \entryItem{Définition de règles de sécurité et de standards dans l’entreprise pour les services AWS}
\end{datedEntry}
\fi

\begin{datedEntry}
    {Avr. -- Juil. 2017}
    {\textbf{Ingénieur logiciel} chez \textbf{AIRBUS DEFENCE AND SPACE} -- Toulouse, France}
    Création d’un outil de visualisation pour modèles de bases de données DynaWorks

    \smallskip
    % \entryItem{Analyse du problème et conception haut-niveau d’une solution}
    \entryItem{Implémentation complète en \textbf{C++ / Qt} basée sur des APIs DynaWorks existantes}
    \entryItem{Optimisation de performances et validation de fiabilité}
    % \entryItem{Documentation utilisateur et design d’interface}
\end{datedEntry}

\begin{datedEntry}
    {Mai -- Août 2015}
    {\textbf{Analyste Programmeur Systèmes} chez \textbf{INFOMIL (E. Leclerc)} -- Toulouse, France}
    \entryItem{Optimisation d’un logiciel \textbf{Perl} de sauvegarde d’équipements réseau}
    \entryItem{Création et implémentation d’une base de données \textbf{MySQL} de gestion de backups}
    % \entryItem{Conception d’un front-end \textbf{CGI} de gestionnaire de configuration de routeurs}
\end{datedEntry}


\section{\faGraduationCap}{Formation}

\begin{datedEntry}
    {2017 -- 2018}
    {\textbf{KTH - Institut Royal de Technologie} -- Stockholm, Suède}
    \textbf{Master en ingénierie informatique}

    \smallskip
    \entryItem{Développement logiciel à grande échelle}
    \entryItem{Sûreté, sécurité et fiabilité du logiciel}
\end{datedEntry}

\begin{datedEntry}
    {2015 -- 2018}
    {\textbf{Grenoble INP -- ENSIMAG} -- Grenoble, France}
    École Nationale Supérieure d’informatique et de Mathématiques Appliquées

    \smallskip
    \textbf{Diplôme d’ingénieur en systèmes d’information}

    \smallskip
    \entryItem{Conception et développement logiciel}
\end{datedEntry}


\section{\faLaptop}{Compétences informatiques}

\titledEntry{Langages}
{Rust, Ruby/Rails, Java/Spring, C++/Qt, Haskell, Clojure, SQL (Postgre), C, Python, ...}

\titledEntry{Outils de développement \& DevOps}
{Git, vim, Maven, Jenkins, Ansible, Artifactory, CircleCI, ...}

\titledEntry{OS}
{Linux (Red Hat / CentOS, Debian, Arch), MS Windows}


\section{\faComment[regular]}{Langues}

\titledEntry{Français}{Langue maternelle}

\titledEntry{Anglais}
{\textbf{TOEIC C1} -- Professionnel, grande aisance à l’écrit et à l’oral}

\titledEntry{Suédois \& Espagnol}
{\textbf{B1 - B2} -- Niveau intermédiaire}

\end{document}